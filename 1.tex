\section{Задание 1. Базис линейного пространства. Координаты вектора.}

\textbf{Условие.}\\
Докажите, что система $A$ является базисом в соответствующем линейном пространстве $L$.
Найдите в этом базисе координаты элемента $x$.

\begin{enumerate}
    \item $L$ - пространство матриц второго порядка\\
    $\displaystyle A = \Set{A_1, A_2, A_3, A_4}\\
    A_1 = \begin{pmatrix}2 & 2 \\ 3 & 1\end{pmatrix}\qquad
    A_2 = \begin{pmatrix}-2 & 3 \\ 4 & 3\end{pmatrix}\qquad
    A_3 = \begin{pmatrix}0 & 1 \\ -1 & 1\end{pmatrix}\qquad
    A_4 = \begin{pmatrix}1 & -3 \\ 2 & -1\end{pmatrix}\\
    x = \begin{pmatrix}-3 & -6 \\ 0 & 2\end{pmatrix}$
    \item $L$ - пространство многочленов степени не больше четырёх, вектор $x = t^4 - t^3 + t^2 - t + 1$\\
    $\displaystyle A = \Set{e_1, e_2, e_3, e_4, e_5}\\
    e_1 = 1 + t\qquad
    e_2 = t + t^2\qquad
    e_3 = t^2 + t^3\qquad
    e_4 = t + t^3\qquad
    e_5 = t^2 + t^4\qquad$
\end{enumerate}
\vspace{10mm}
\noindent\textbf{Решение.}\\
\begin{enumerate}
    \item $L$ - пространство матриц второго порядка\\
    $\displaystyle A = \Set{A_1, A_2, A_3, A_4}\\
    A_1 = \begin{pmatrix}2 & 2 \\ 3 & 1\end{pmatrix}\qquad
    A_2 = \begin{pmatrix}-2 & 3 \\ 4 & 3\end{pmatrix}\qquad
    A_3 = \begin{pmatrix}0 & 1 \\ -1 & 1\end{pmatrix}\qquad
    A_4 = \begin{pmatrix}1 & -3 \\ 2 & -1\end{pmatrix}\\
    x = \begin{pmatrix}-3 & -6 \\ 0 & 2\end{pmatrix}$
    \item $\forall P_n \ (n \leq 4) \ \exists! (a, b, c, d, e) : P(x) = a + bx + cx^2 + dx^3 + ex^4$ -
    существует биекция между пространством полиномов степени $\leq 4$ и векторов размерностью 5.\\
    Сопоставим каждому полиному $e_1 \mathellipsis e_5$ вектора $V_1 \mathellipsis V_5$:\\
    \begin{align*}
    e_1 = 1 + t \quad &\implies \quad V_1 = (1; 1; 0; 0; 0)\\
    e_2 = t + t^2 \quad &\implies \quad V_2 = (0; 1; 1; 0; 0)\\
    e_3 = t^2 + t^3 \quad &\implies \quad V_3 = (0; 0; 1; 1; 0)\\
    e_4 = t + t^3 \quad &\implies \quad V_4 = (0; 1; 0; 1; 0)\\
    e_5 = t^2 + t^4 \quad &\implies \quad V_5 = (0; 0; 1; 0; 1)\\
    \end{align*}

    $\displaystyle V = \Set{V_i}_{i = 1..5} =
    \begin{pmatrix}1 & 0 & 0 & 0 & 0 \\ 1 & 1 & 0 & 1 & 0 \\ 0 & 1 & 1 & 0 & 1 \\ 0 & 0 & 1 & 1 & 0 \\ 0 & 0 & 0 & 0 & 1\end{pmatrix}
    \quad\sim\quad
    \begin{pmatrix}1 & 0 & 0 & 0 & 0 \\ 0 & 1 & 0 & 1 & 0 \\ 0 & 1 & 1 & 0 & 1 \\ 0 & 0 & 1 & 1 & 0 \\ 0 & 0 & 0 & 0 & 1\end{pmatrix}
    \quad\sim\quad
    \begin{pmatrix}1 & 0 & 0 & 0 & 0 \\ 0 & 1 & 0 & 1 & 0 \\ 0 & 0 & 1 & -1 & 1 \\ 0 & 0 & 0 & 2 & -1 \\ 0 & 0 & 0 & 0 & 1\end{pmatrix}$
    \quad-\quad ступенчатый вид

    $\begin{pmatrix}1 & 0 & 0 & 0 & 0 \\ 0 & 1 & 0 & 0 & 0 \\ 0 & 0 & 1 & 0 & 0 \\ 0 & 0 & 0 & 1 & 0 \\ 0 & 0 & 0 & 0 & 1\end{pmatrix}$
    \quad-\quad канонический вид

    Ступенчатый и канонические виды матрицы не имеют нулевых строк, поэтому система $V$ линейно зависима,
    а так как между $V$ и $A$ существует биекция, то и $A$ линейно зависима и образует базис

    Представим $x$ в базисе $V$ как $y$:

    $x = t^4 - t^3 + t^2 - t + 1 = x_1 e_1 + x_2 e_2 + x_3 e_3 + x_4 e_4 + x_5 e_5 \quad\sim\quad y = \begin{pmatrix}1 \\ -1 \\ 1 \\ -1 \\ 1\end{pmatrix}$\\
    $(V|y) = \begin{pmatrix}1 & 0 & 0 & 0 & 0 \\ 1 & 1 & 0 & 1 & 0 \\ 0 & 1 & 1 & 0 & 1 \\ 0 & 0 & 1 & 1 & 0 \\ 0 & 0 & 0 & 0 & 1\end{pmatrix}\begin{pmatrix}1 \\ -1 \\ 1 \\ -1 \\ 1\end{pmatrix}
    \quad\sim\quad
    \begin{pmatrix}1 & 0 & 0 & 0 & 0 \\ 0 & 1 & 0 & 1 & 0 \\ 0 & 1 & 1 & 0 & 1 \\ 0 & 0 & 1 & 1 & 0 \\ 0 & 0 & 0 & 0 & 1\end{pmatrix}\begin{pmatrix}1 \\ -2 \\ 1 \\ -1 \\ 1\end{pmatrix}
    \quad\sim\quad
    \begin{pmatrix}1 & 0 & 0 & 0 & 0 \\ 0 & 1 & 0 & 1 & 0 \\ 0 & 0 & 1 & -1 & 1 \\ 0 & 0 & 0 & 2 & -1 \\ 0 & 0 & 0 & 0 & 1\end{pmatrix}\begin{pmatrix}1 \\ -2 \\ 3 \\ -4 \\ 1\end{pmatrix}$

    $\displaystyle x_5 = \frac{1}{1} = 1;\\
    2x_4 - x_5 = -4 \implies x_4 = -\frac{3}{2}\\
    x_3 - x_4 + x_5 = 3 \implies x_3 = \frac{1}{2}\\
    x_2 + x_4 = -2 \implies x_2 = -\frac{1}{2}\\
    x_1 = 1$

    Таким образом, в базисе $A$ $\displaystyle x = e_1 - \frac{e_2}{2} + \frac{e_3}{2} - \frac{3e_4}{2} + e_5$

    \textit{Ответ}: $A$ - это базис, в базисе $A$ $\displaystyle x = e_1 - \frac{e_2}{2} + \frac{e_3}{2} - \frac{3e_4}{2} + e_5$.
\end{enumerate}

\clearpage